\documentclass{article}
\usepackage{latexsym}
\usepackage{epsfig}
\usepackage{latexsym}
\usepackage{amssymb,amsmath,amsthm}
\usepackage[noend]{algorithmic}
\usepackage{algorithm}
%\usepackage{colortbl}
\usepackage{tikz}
\usetikzlibrary{arrows,decorations,decorations.shapes,backgrounds,shapes}

\usepackage{color}
\usepackage{xcolor}
%\usepackage{listings}

%\usepackage{caption}
%\DeclareCaptionFont{white}{\color{white}}
%\DeclareCaptionFormat{listing}{\colorbox{gray}{\parbox{\textwidth}{#1#2#3}}}
%\captionsetup[lstlisting]{format=listing,labelfont=white,textfont=white}
\usepackage{mathtools}

\renewcommand{\algorithmicrequire}{\textbf{Input:}}
\renewcommand{\algorithmicensure}{\textbf{Output:}}

% Some more house keeping
\DeclarePairedDelimiter\ceil{\lceil}{\rceil}
\DeclarePairedDelimiter\floor{\lfloor}{\rfloor}

\begin{document}

\title{CIS507: Design \& Analysis of Algorithms\\\emph{Homework 1 (with answers), Spring 2014}}


%\date{DRAFT!}
\date{March 5th, 2014}

\maketitle
\textbf{Student Name: Yanan Xiao}

\vspace{0.2in}

\textbf{Student ID\@: 131102019}

\vspace{0.2in}


\noindent{\bf Q1. }
Give asymptotic upper bounds for $T(n)$ for each of the following
recurrences  (use the $O$-notation). Assume that $T(n)$ is a
non-negative constant for $n$ sufficiently large (in terms of
$\alpha$). Make your bounds as tight as possible, and justify your
answers. 

\begin{enumerate}
\item  \textbf{(1 point)} $T(n)=n^{1-\alpha}\cdot
  T(n^{\alpha})+\Theta(n)$, for a constant $\alpha\in (0,1)$.   
\item  \textbf{(1 points)} $T(n)= T(n-1)+T(\alpha\cdot n)+1$, for a
  constant $\alpha\in (0,1)$. 
\end{enumerate}
\noindent \underline{\textbf{ANSWER:}}

\begin{enumerate}
\item
$$ T(n)=n^{1-\alpha}\cdot T(n^{\alpha})+\Theta(n) \Leftrightarrow
\frac{T(n)}{n} = \frac{T(n^{\alpha})}{n^{\alpha}} + \Theta(1) $$ 
Let
$$ \frac{T(n)}{n} = P(n) \Rightarrow \frac{T(n)}{n} =
\frac{T(n^{\alpha})}{n^{\alpha}} + \Theta(1) \Leftrightarrow P(n) =
P(n^{\alpha}) + \Theta(1) $$  
Let $$ n = 2^m \Leftrightarrow m = \log n \Rightarrow S(2^m) =
S(2^{\alpha m}) + \Theta(1) $$ 
Let
$$ Q(m) = P(2^m) \Rightarrow Q(m) = Q(\alpha m) + \Theta(1) $$ 
By Master Theorem, we can get $$ m^{\log_b a} = m^{\log_{\frac{1}{\alpha}}
  1} = m^0 = 1 \Rightarrow Q(m) = \Theta(\log m) \Rightarrow
P(n)=\Theta(\log \log n) $$ 
But $$ T(n) = nP(n) \Rightarrow T(n) = \Theta(n\log \log n) = O(n \log
\log n)$$ 

\item
From the question description we can see that $T(n)$ is sufficiently
large in terms of $\alpha$.\\
To be specific, let's assume $\alpha=0.5$, then we can rewrite the
recurrence equation as $T(n) = T(n-1) + T(0.5n) + 1$. By using tree
induction it is not hard to guess $T(n) = O(n\log n)$. The proof is as
follows. 



\end{enumerate}


\medskip 

\noindent{\bf Q2. }
\textbf{(2 points)}
Consider the following problem called MAXCUT: given an undirected
graph $G=(V,E)$ with non-negative edge weights $w_e$ for $e\in E$,
find a partition $(S,V\setminus S)$ of the vertices that maximizes the
total weight of the edges crossing the cut, that is,
$\sum_{e\in\delta(S)}w_{e}$, where $\delta(S)$ is the set of edges
that have one end-point in $S$ and another in $V\setminus S$.     

Consider the following randomized algorithm: Select a subset $S$ by
picking each vertex in $V$ independently with probability
$\frac{1}{2}$. Show that the expected wight of the edges in the cut
$(S,V\setminus S)$ is a factor of $\frac{1}{2}$ of the total weight,
that is: $$\mathbb{E}\left[\sum_{e\in\delta(S)}w_{e}\right]=
\frac{1}{2}\sum_{e\in E}w_e$$ ({\it Hint:} use an indicator random
variable for each edge.)\\  
\noindent \underline{\textbf{ANSWER:}}
Let's define an indicator random variable $P_{e}$ where
\begin{equation}
  \label{eq:indicator-random-variable}
  P_{e} = 
  \begin{cases}
    1 & the~edge~crosses~the~cut\\
    0 & the~edge~does~not~cross~the~cut
  \end{cases}
\end{equation}
Since each vertex is picked up independently with a $0.5$ probability,
therefore the probability for a vertex's both ends are in the same set
is $0.5$. Thus we have $\mathbb{E}[P_{e}]~=~0.5$.
$$\sum_{e\in\delta(S)}w_{e}~=~\sum_{e\in E}w_{e}P_{e}$$
Take the expectation of both sides, we have 
$$E[\sum_{e\in\delta(S)}w_{e}]~=~E[\sum_{e\in E}w_{e}P_{e}]$$ 
With the linearity of expectation it's not hard to get
$$E[\sum_{e\in\delta(S)}w_{e}]~=~\sum_{e\in
  E}w_{e}\mathbb{E}[P_{e}]~=~\frac{1}{2}\sum_{e\in E}w_{e}$$ 
\medskip

\noindent{\bf Q3.}
Suppose that we would like to analyze the change in price for a given
stock. We observe the different prices over a period of $n$ days. Let
$A[i]$ be the observed price in day $i$. We would like to compute: 
\begin{enumerate}
\item[(I)]  the smallest absolute price difference: $\min_{1\le i,j\le
    n,i\ne j}|A[i]-A[j]|$;  
\item[(II)] the largest absolute price difference: $\max_{1\le i,j\le
    n}|A[i]-A[j]|$;  
\item[(III)]  the average absolute price difference:
  $\frac{1}{n(n-1)}\sum_{1\le i,j\le n}|A[i]-A[j]|$;  
\item[(IV)]  the median absolute price difference:
  $\text{median}(\{|A[i]-A[j]|:~1\le i,j\le n\})$.  
\end{enumerate}
\begin{enumerate}
\item[(i)] \textbf{(1 point)} give an $O(n^2)$ deterministic algorithm
  for computing (I), (II), (III) and (IV); 
\item[(ii)] \textbf{(1 point)} give an $O(n\log n)$ deterministic
  algorithm for computing (I);  
\item[(iii)] \textbf{(1 point)} give an $O(n)$ deterministic algorithm
  for computing (II);  
\item[(iv)] \textbf{(1 point)} give an $O(n\log n)$ deterministic
  algorithm for computing (III);  
\item[(v)] \textbf{(1 point)}  give a randomized algorithm with
  $O(n^2)$ expected running time for computing (IV). 
\end{enumerate}
Implement the four algorithms in (ii), (iii), (iv) and (v). For
testing purposes, your program  should accept as an input a file
``test.in'', containing $n$, followed by the set of $n$ numbers (1 per
line). It should output the four values described in (I), (II), (III),
and (IV).\\
\noindent \underline{\textbf{ANSWER:}}
In the following solutions we will use two auxiliary algorithms. One
is standard merge sort, the other is a randomized selection algorithm.
\par
\begin{figure*}[!htbp]
  \centering
  \begin{algorithmic}
    \REQUIRE An array $A$ of length $n$
    \ENSURE The smallest absolute difference
    \FOR{$i = 1$ to $n$}
	\FOR{$j = i + 1$ to $n$}
		\STATE $B[k] = |A[i] - A[j]|$
	\ENDFOR
    \ENDFOR
    \STATE $min = B[1]$
    \FOR{$i = 1$ to $len(B)$}
	\IF{$B[i] < min$}
		\STATE $min = B[i]$
	\ENDIF
    \ENDFOR
    \STATE return $min$

  \end{algorithmic}
  \caption{Deterministic Algo for Smallest Abs}
  \label{fig:deter-smallest}
\end{figure*}

\begin{figure*}[!htbp]
  \centering
  \begin{algorithmic}
    \REQUIRE An array $A$ of length $n$
\ENSURE The largest absolute difference
\FOR{$i = 1$ to $n$}
	\FOR{$j = i + 1$ to $n$}
		\STATE $B[k] = |A[i] - A[j]|$
	\ENDFOR
\ENDFOR
\STATE $max = B[1]$
\FOR{$i = 1$ to $len(B)$}
	\IF{$B[i] > max$}
		\STATE $max = B[i]$
	\ENDIF
\ENDFOR
\STATE return $max$
  \end{algorithmic}
  \caption{Deterministic Algo for Largest Abs}
  \label{fig:deter-largest}
\end{figure*}

\begin{figure*}[!htbp]
  \centering
  \begin{algorithmic}
    \REQUIRE An array $A$ of length $n$
\ENSURE The average absolute difference
\STATE $C = 0$
\FOR{$i = 1$ to $n$}
	\FOR{$j = i + 1$ to $n$}
		\STATE $B[k] = |A[i] - A[j]|$
		\STATE $C = C + B[k]$
	\ENDFOR
\ENDFOR
\STATE return $\frac{C}{n(n-1)}$
  \end{algorithmic}
  \caption{Deterministic Algo for Avg Abs}
  \label{fig:deter-avg}
\end{figure*}

\begin{figure*}[!htbp]
  \centering
  \begin{algorithmic}
    \REQUIRE An array $A$ of length $n$
\ENSURE The median absolute difference
\FOR{$i = 1$ to $n$}
	\FOR{$j = i + 1$ to $n$}
		\STATE $B[k] = |A[i] - A[j]|$
	\ENDFOR
\ENDFOR
\STATE $C$ = Merge-sort($B$)
\STATE return $C[mid]$
  \end{algorithmic}
  \caption{Deterministic Algo for Median Abs}
  \label{fig:deter-median}
\end{figure*}

\begin{figure*}[!htbp]
  \centering
  \begin{algorithmic}
    \REQUIRE An array $A$ of length $n$
\ENSURE The smallest absolute difference
\STATE $B$ = Merge-sort($A$)
\STATE $c = B[2]-B[1]$
\FOR{$i = 1$ to $n-1$}
	\STATE $d = B[i+1] - B[i]$
	\IF{$d < c$}
		\STATE $c = d$
	\ENDIF
\ENDFOR
\STATE return $c$
  \end{algorithmic}
  \caption{Deterministic Algo for Smallest Abs with $O(n\log n)$}
  \label{fig:deter-smallest-1}
\end{figure*}

\begin{figure*}[!htbp]
  \centering
  \begin{algorithmic}
    \REQUIRE An array $A$ of length $n$
\ENSURE The largest absolute difference
\STATE $min = A[1], max = A[1]$
\FOR{$i = 1$ to $n$}
	\IF{$A[i] < min$}
		\STATE $min = A[i]$
	\ENDIF
	\IF{$A[i] > max$}
		\STATE $max = A[i]$
	\ENDIF
\ENDFOR
\STATE return $max - min$
  \end{algorithmic}
  \caption{Deterministic Algo for Largest Abs with $O(n)$}
  \label{fig:deter-largest-1}
\end{figure*}

\begin{figure*}[!htbp]
  \centering
  \begin{algorithmic}
    \REQUIRE An array $A$ of length $n$
\ENSURE The average absolute difference
\STATE $B =$ Merge-sort ($A$)
\STATE $sum = 0$
\FOR{$i = 1$ to $n$}
	\STATE $sum = sum + B[i] * ( 2*i - len(B) + 1 ) $
\ENDFOR
\STATE return $\frac{sum}{n(n-1)}$
  \end{algorithmic}
  \caption{Deterministic Algo for Avg Abs with $O(n\log n)$}
  \label{fig:deter-avg-1}
\end{figure*}

\begin{figure*}[!htbp]
  \centering
  \begin{algorithmic}
        \REQUIRE An array $A$ of length $n$
    \ENSURE The median absolute difference
    \FOR{$i = 1$ to $n$}
	\FOR{$j = i + 1$ to $n$}
		\STATE $B[k] = |A[i] - A[j]|$
	\ENDFOR
    \ENDFOR
    \STATE $C$= Randomized-select $(A, 1, n, \floor{n/2})$
    \STATE return $C$
  \end{algorithmic}
  \caption{Randomized Algo for Median Abs with $O(n^{2})$}
  \label{fig:random-median}
\end{figure*}









\medskip 

\noindent{\bf Q4. (4 points)}
Implement a {\it perfect} hash table, where keys are decimal numbers,
each having at most 10 digits. For both hash levels, use the class of
universal hash functions of the dot-product form: if the hash table
size is a prime $m$, pick a random sequence $\mathbf{a}:=\langle
a_0,a_1,\ldots,a_9\rangle$, where each $a_i\in\{0,1\ldots,m-1\}$;
given a key $k$, decompose it into a sequence of decimal digits
$\mathbf k:=\langle k_0,k_1,\ldots,k_9\rangle$, then use hash
functions of the form $ h_{\mathbf{a}}(k)=(\sum_{r=0}^9a_ik_i)\mod m$.   
Your table should have  no collisions, and and uses at most $8n$ table
entries, in total. ({\it Hint:} use the fact that for any positive
integer $n$, there is at least one prime between $n$ and $2n$.) 

For testing purposes, your program  should accept as an input a file
"test.in" containing the number of keys $n$, followed by the set of
keys to be hashed (1 per line). It should output in another file
"test.out", the following lines: the first line (call it line 0)
contains the values chosen for the first-level hash function in the
following order (separated by spaces): $m, a_0,a_1,\ldots,a_r$; then
for $i=1,\ldots,m$, the $i$th line contains the values corresponding
to the second level-hash function chosen at the $i$th row in the first
level table (again in the order $m(i), a_0(i),a_1(i),\ldots,a_r(i)$;
output "0 0" if that row is empty). Following this, the file should
contain triples (one per line): $(k,h(k),h_i(k))$, where $k$ is the
key, $i=h(k)$ is the index in the first-level hash table, $h_i(k)$ is
the index in the second level hash table. 
%%%%%%%%%%%%%%%%%%%%%%%%%%%%%%%%%%%%%%%%%%%%%%
\end{document}
