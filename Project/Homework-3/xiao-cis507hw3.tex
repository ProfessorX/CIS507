\documentclass{article}
\usepackage{latexsym}
\usepackage{epsfig}
\usepackage{latexsym}
\usepackage{amssymb,amsmath,amsthm}

%\usepackage{colortbl}
\usepackage{tikz}
\usetikzlibrary{arrows,decorations,decorations.shapes,backgrounds,shapes}

\usepackage{color}
\usepackage{xcolor}
%\usepackage{listings}

%\usepackage{caption}
%\DeclareCaptionFont{white}{\color{white}}
%\DeclareCaptionFormat{listing}{\colorbox{gray}{\parbox{\textwidth}{#1#2#3}}}
%\captionsetup[lstlisting]{format=listing,labelfont=white,textfont=white}

\begin{document}

\title{CIS507: Design \& Analysis of Algorithms\\\emph{Homework 3,
    Spring 2014} (With answers)} 


%\date{DRAFT!}
\date{}

\maketitle

\noindent{\bf Q1. {\bf (2 points) }  Exercise 24.4--8 in the text
  book.} 
Let $Ax\le b$ be a system of $m$ difference constraints in $n$ unknowns.
 Show that the Bellman-Ford algorithm, when run on the corresponding
 constraint graph, maximizes $\sum_{i=1}^nx_i$ subject to $Ax\le b$
 and $x_i\leq 0$ for all $x_{i}$. 

\medskip
\noindent \underline{\textbf{ANSWER:}}
\begin{enumerate}
\item Start with proving $x_{i}\leq 0$ for all $x_{i}$.\\
  We have $d[v_{0}]=0$ and $w(v_{0},v_{i})=0$ in the beginning in
  constraint graph. After the ``relaxation'' step in Bellman-Ford
  algorithm, the shortest path $\delta(v_{0},v_{i})\leq 0$ because
  ``relaxation'' step only makes the path shorter (or at least the
  same). According to Theorem 24.9, $x_{i}=\delta(v_{0},v_{i})$ is a
  feasible solution thus $x_{i}\leq 0$ for all $x_{i}$.
\item Life is awesome in the United Arab Emirates!

\end{enumerate}


\medskip

\noindent{\bf Q2. } We are given a set of jobs $\mathcal{J}$ to be
scheduled on an unlimited number of machines. Each job $j\in
\mathcal{J}$ has a processing time $p_j> 0$. There is also a set of
{\it precedence constraints} describing, for each job $j$, the set of
jobs that have to be completed before starting job $j$. We would like
to find, for each job $j$ the {\it earliest scheduling time} $S_j$,
that is, the first point in time at which the job can be started
without violating any precedence constraints. 

\begin{enumerate}
\item[(I)]  {\bf (1.0 point).} Model this as a graph problem and give
  a sufficient and necessary condition for the existence of a feasible
  schedule. 
\item[(II)]  {\bf (2.0 point).} Give an efficient algorithm for either
  finding the earliest scheduling times $S_j$, for $j\in\mathcal{J}$,
  or declaring that no feasible schedule exists. Analyze the running
  time of the algorithm. 
\item[(III)] {\bf (2.0 point).} Suppose now that, instead of
  processing times, we are given for some pairs of jobs $(i,j)$ the
  {\it time delay} $t_{ij}\in\mathbb{R}$ (which can be {\it
    negative}),  imposing that $S_j\ge S_i+t_{ij}$, that is, job $j$
  can only be started after $t_{ij}$ units of time from the start time
  of job $i$. Model this as a graph problem and give an efficient
  algorithm for either finding the earliest scheduling times $S_j$,
  for $j\in\mathcal{J}$, or declaring that no feasible schedule
  exists. Analyze the running time of the algorithm. 
\item[(IV)]  {\bf (3 points).} Implement both algorithms in (II) and
  (III). For testing purposes, your program in (II) should accept as
  an input a file "test.in" containing the number $n$ of jobs,
  followed by a list of $n$ lines; line $j$ contains $p_j$ followed by
  the list of jobs that have to be completed before starting job $j$
  (separated by spaces); your program in (III) should accept as an
  input a file "test.in" containing the number $n$ of jobs, followed
  by a list of $n$ lines; line $j$ contains pairs $(i,t_{ij})$
  (separated by spaces).  The two programs should output in another
  file "test.out" either the earliest scheduling times
  $S_1,\ldots,S_n$ (separated by spaces), or the word "INFEASIBLE", if
  there is no feasible schedule.  
\end{enumerate}

\medskip

\noindent{\bf Q3.}
\begin{enumerate}
\item {\bf (1 point).} Several families go out to dinner together. To
  increase their social interaction, they would like to sit at tables
  so that no two members of the same family are at the same
  table. Show how to formulate finding a seating arrangement that
  meets this objective as a maximum flow problem. Assume that the
  dinner contingent has p families and that the $i$th family has
  $a(i)$ members. Also assume that $q$ tables are available and that
  the $j$th table has a seating capacity of $b(j)$. 
\item {\bf (2 points). Exercise 26.3-5 in the text book.} We say that
  a bipartite graph $G=(V,E)$, where $V=L\cup R$, is $d$-regular if
  every vertex $v\in V$ has degree exactly $d$. Every $d$-regular
  bipartite graph has $|L| = |R|$. Prove that every $d$-regular
  bipartite graph has a matching of cardinality $|L|$ by arguing that
  a minimum cut of the corresponding flow network has capacity $|L|$.
\end{enumerate}

\end{document}
